\documentclass{article}
\usepackage{import}
\usepackage{example}
\usepackage{Estilo_mio}
\usepackage{siunitx}
\usepackage{steinmetz}
\usepackage{enumerate}
\usepackage{verbatim}

\headerp{Cátedra de Síntesis de Redes Activas - Trabajos Prácticos de Laboratorio}
 
\titulo{Trabajo Práctico de Laboratorio Nº1}%\\
%AO Ideal: Circuitos Analógicos Lineales y No Lineales.}
\data{Año 2021}
% \autor{nombre}
\author{
\large
\textsc{\textbf{Alumno:} Juan Perez 30.000.000 JP@gmail.com}\\
-\\
\textsc{\textbf{Profesor Titular:} Dr. Ing. Pablo Ferreyra}\\
\textsc{\textbf{Profesor Adjunto:} Ing. César Reale} \\
\textsc{\textbf{Profesor Ayudate:} Ing. Daniel Sánchez} \\
\textsc{\textbf{Ayudante alumno:} Lucas Heraldo Duarte} \\
\normalsize Facultad de Ciencias Exactas Físicas y Naturales \\ Universidad Nacional de Córdoba  % Your institution
\vspace{-5mm}
}


\begin{document}

\maketitle

\begin{center}

\textbf{Síntesis de Redes Activas}\\

\textbf{Ingeniería Electrónica - Agosto 2021}\\

\end{center}

\thispagestyle{fancy}

%%%%%%%%%%%%%%%%%%%%%%%%%%%%%%%%%%%%%%%%%%%%%%%%%%

 \begin{abstract}

 \noindent RESUMEN .

 \end{abstract}

\section{Metodología general}
\import{}{0-Metodologia}

\newpage{}
\section{Circuito I: Amplificador diferencial}
\import{}{1-CI}


\end{document}

